\documentclass[twoside]{article}
\usepackage{parskip}
\begin{document}

  \title{Purification Paper Worksheet} \author{UCSF Bootcamp 2014}
  \maketitle
  For this project, you will work in groups of 3 or 4 to analyze and discuss a paper which centers around some interesting molecular biology approaches. On Friday, you will give a 15-20 minute presentation on some of the techniques used in the paper. You can choose the 2-3 techniques which interest you most, rather than trying to cover every technique in the paper.

  Each of the papers that you are reading focus on expressing, purifying and analyzing a protein or complex. These are a few guiding questions to help with preparing your presentations. You can frame the answers however you like; this is not meant to be a checklist. We suggest that you spend a third to a half of your presentation discussing the final two questions.
  \begin{enumerate}
    
    \item What biological compound is being studied in this paper?
    \item What techniques did the authors use to express and purify the compound?
    \item Why did the authors choose to use these techniques? In other words, what makes this purification strategy preferable? 
    \item What experimental assays did the authors perform with their purified compound?
    \item What new information did the authors gain from these analyses?
    \item What are some downsides of this overall experimental approach? What are the limitations of the techniques being used?
    \item What would be a good alternative workflow?
    \item As a though experiment, design an experiment using some or all of the techniques the authors used to express and purify a protein that you are interested in.
    \item What experimental assays would you perform on your newly purified protein?

  \end{enumerate}

\end{document}